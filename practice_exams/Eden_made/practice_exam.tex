\documentclass{article}
\usepackage{enumitem}
\begin{document}
\section*{Questions}
\textbf{Information for Questions 1--2}\\
We observe data $y=(12.5,\,15.2,\,14.8,\,13.9,\,16.1)$.

\begin{enumerate}
\item The sample mean $\bar y$ is closest to
    \begin{enumerate}[label=\Alph*.]
        \item 13.9
        \item 14.5
        \item 15.2
        \item 12.5
        \item 16.1
    \end{enumerate}
\item The sample standard deviation $s$ is closest to
    \begin{enumerate}[label=\Alph*.]
        \item 0.9
        \item 1.1
        \item 1.4
        \item 2.0
        \item 3.0
    \end{enumerate}
\item While browsing past marathon results, we notice a finishing time of $2{:}07{:}30$ in one race. Using only this information, select the best option:
    \begin{enumerate}[label=\Alph*.]
        \item This must be an unusually fast time.
        \item Since it’s one race, this time is typical for marathons.
        \item We can’t tell if it’s unusual without knowing the distribution of marathon times.
        \item It is normal that all elite marathons finish under 2:05.
        \item Every marathon winner finishes at exactly 2:07{:}30.
    \end{enumerate}
\item Are variation and uncertainty important concepts in statistics?
    \begin{enumerate}[label=\Alph*.]
        \item Yes—models describe variation, and we quantify uncertainty wherever possible.
        \item No—statistics finds exact population parameters without error.
        \item Yes—but only to make analysis look sophisticated.
        \item No—big data eliminates uncertainty.
        \item No—once data are collected, uncertainty disappears.
    \end{enumerate}

\textbf{Information for Questions 5--7}\\
A discrete variable $Y$ takes values $\{1,2,5\}$ with probabilities $0.20,\,0.50,\,0.30$ respectively.
\item $\Pr(Y\ge 2)$ equals
    \begin{enumerate}[label=\Alph*.]
        \item 0.20
        \item 0.30
        \item 0.50
        \item 0.80
        \item 1.00
    \end{enumerate}
\item $E[Y]$ is
    \begin{enumerate}[label=\Alph*.]
        \item 2.1
        \item 2.5
        \item 2.7
        \item 3.0
        \item 3.2
    \end{enumerate}
\item Which best describes $E[Y]$?
    \begin{enumerate}[label=\Alph*.]
        \item A randomly selected person will have exactly $E[Y]$.
        \item The expected/average value in repeated sampling from this distribution.
        \item Half the population will be below $E[Y]$.
        \item Having exactly $E[Y]$ implies increased risk.
        \item $Y$ is normally distributed with mean $E[Y]$.
    \end{enumerate}

\textbf{Information for Questions 8--11}\\
Let $A$ be ``a customer is a subscriber'' and $B$ be ``they purchase a premium plan''. Suppose $\Pr(A)=0.40$, $\Pr(B\mid A)=0.35$, $\Pr(B\mid A^c)=0.10$.
\item Best interpretation of $\Pr(B\mid A^c)$:
    \begin{enumerate}[label=\Alph*.]
        \item Probability a non-subscriber purchases the premium plan.
        \item Probability someone is not a subscriber given they purchased.
        \item Probability a subscriber purchases the premium plan.
        \item Probability someone is not a subscriber and purchases.
        \item Probability of purchasing regardless of subscription.
    \end{enumerate}
\item $\Pr(A\mid B)$ is closest to
    \begin{enumerate}[label=\Alph*.]
        \item 0.30
        \item 0.50
        \item 0.60
        \item 0.70
        \item 0.90
    \end{enumerate}
\item $\Pr(A\ \text{or}\ B)$ is closest to
    \begin{enumerate}[label=\Alph*.]
        \item 0.20
        \item 0.40
        \item 0.46
        \item 0.55
        \item 1.00
    \end{enumerate}
\item $\Pr(B)$ is closest to
    \begin{enumerate}[label=\Alph*.]
        \item 0.10
        \item 0.20
        \item 0.30
        \item 0.35
        \item 0.50
    \end{enumerate}
\item What is the best description of a random variable?
    \begin{enumerate}[label=\Alph*.]
        \item A summary of both population and sample.
        \item A variable that must be normal.
        \item A random process with a numerical outcome.
        \item A fixed but unknown value.
        \item A value chosen by the analyst.
    \end{enumerate}
\item Which is best modelled as continuous?
    \begin{enumerate}[label=\Alph*.]
        \item Number of emails received today.
        \item Count of customers who clicked ``Buy''.
        \item CO$_2$ concentration in air (ppm).
        \item Number of red cards in a match.
        \item Count of goals in a game.
    \end{enumerate}

\textbf{Information for Questions 14--15}\\
Let $X$ and $Y$ be independent with $E[X]=8$, $\mathrm{Var}(X)=1.2$ and $E[Y]=5$, $\mathrm{Var}(Y)=0.8$. Define the score $S=3X-2Y$.
\item $E[S]$ is
    \begin{enumerate}[label=\Alph*.]
        \item 6
        \item 10
        \item 12
        \item 14
        \item 16
    \end{enumerate}
\item The standard deviation of $S$, $sd(S)$, is closest to
    \begin{enumerate}[label=\Alph*.]
        \item 1.10
        \item 2.24
        \item 3.74
        \item 4.50
        \item 6.00
    \end{enumerate}

\textbf{Information for Questions 16--19}\\
A test score $Y\sim \mathcal N(\mu=70,\ \sigma=12)$ for healthy adults.
\item A person has $z=-0.75$. Their score is closest to
    \begin{enumerate}[label=\Alph*.]
        \item 52.0
        \item 58.0
        \item 61.0
        \item 70.0
        \item 79.0
    \end{enumerate}
\item Which R command computes $\Pr(Y>82)$?
    \begin{enumerate}[label=\Alph*.]
        \item 1 - pnorm(82)
        \item 1 - pnorm(1.0)
        \item pnorm(82)
        \item pnorm(1.0)
        \item 1 - pnorm(-1.0)
    \end{enumerate}
\item If $n=100$ independent adults are sampled, the sampling distribution of $\bar Y$ is
    \begin{enumerate}[label=\Alph*.]
        \item Normal, mean 70, sd 12
        \item Normal, mean 70, sd 1.2
        \item Normal, mean 0, sd 1
        \item $t$ with 99 df
        \item Unknown
    \end{enumerate}
\item If the population is not normal but not extremely skewed (still $\mu=70,\sigma=12$), for $n=100$ the sampling distribution of $\bar Y$ is approximately
    \begin{enumerate}[label=\Alph*.]
        \item Normal, mean 70, sd 12
        \item Normal, mean 70, sd 1.2
        \item Normal, mean 0, sd 1
        \item $t$ with 99 df
        \item Unknown
    \end{enumerate}

\textbf{Information for Questions 20--22}\\
A researcher measures a continuous outcome with unknown $\sigma$. Sample size $n=25$. Data look roughly normal.
\item A 95\% CI for $\mu$ should be
    \begin{enumerate}[label=\Alph*.]
        \item $\hat p \pm z_{0.975}\sqrt{\hat p(1-\hat p)/n}$
        \item $\bar y \pm z_{0.975}\,\sigma/\sqrt n$
        \item $\mu \pm z_{0.975}\,\sigma/\sqrt n$
        \item $\bar y \pm z_{0.975}\, s/\sqrt n$
        \item $\bar y \pm t_{24,0.975}\, s/\sqrt n$
    \end{enumerate}
\item If we raise confidence from 95\% to 99\%, what changes?
    \begin{enumerate}[label=\Alph*.]
        \item Estimate changes; unclear effect on width
        \item Standard error changes; interval wider
        \item Standard error changes; interval narrower
        \item Multiplier changes; interval wider
        \item Multiplier changes; interval narrower
    \end{enumerate}
\item A 99\% CI for $\mu$ is $(3.54,\ 3.82)$. Best interpretation:
    \begin{enumerate}[label=\Alph*.]
        \item 99\% of individual observations lie in $(3.54,3.82)$.
        \item A random future observation has 99\% chance to lie in $(3.54,3.82)$.
        \item There is probability 0.99 that the fixed $\mu$ is in $(3.54,3.82)$.
        \item 99\% of sample means fall in $(3.54,3.82)$.
        \item We are 99\% confident the true mean lies in $(3.54,3.82)$.
    \end{enumerate}
\item For $H_0:\mu=\mu_0$ vs $H_A:\mu\neq \mu_0$, with $\alpha=0.01$ and $p=10^{-6}$:
    \begin{enumerate}[label=\Alph*.]
        \item Very small $p$ means a very large effect size must exist.
        \item We can be certain $\mu\neq\mu_0$.
        \item Data would be very unlikely if $H_0$ were true.
        \item $\Pr(\mu=\mu_0)=10^{-6}$.
        \item $\Pr(\text{our calculations are correct})=10^{-6}$.
    \end{enumerate}
\item Which best describes estimation?
    \begin{enumerate}[label=\Alph*.]
        \item Assume population mean equals sample mean.
        \item Compute SE as $s/\sqrt n$.
        \item Use population data to compute a statistic.
        \item Guess data values to improve model fit.
        \item Use a statistic to make an informed guess about an unknown parameter.
    \end{enumerate}

\textbf{Information for Questions 25--26}\\
The R object \texttt{otter} has variables \texttt{weight} (kg) and \texttt{length} (cm).
\item What does \texttt{mean(otter$weight)} compute?
    \begin{enumerate}[label=\Alph*.]
        \item Sample mean of length
        \item Sample sd of length
        \item Sample mean of weight
        \item Sample sd of weight
        \item Median weight
    \end{enumerate}
\item What does \texttt{sd(otter$length)} compute?
    \begin{enumerate}[label=\Alph*.]
        \item Sample mean of length
        \item Sample sd of length
        \item Sample mean of weight
        \item Sample sd of weight
        \item Median length
    \end{enumerate}

\textbf{Information for Questions 27--30 (Paired data)}\\
Thirty patients’ stress hormone levels are recorded before and after mindfulness training. R output:

\texttt{t = -2.57, df = 29, p-value = 0.012}\\
\texttt{95 percent CI for mean difference (after - before): (-1.15, -0.15)}\\
\texttt{mean difference = -0.65}
\item With $\alpha=0.05$, best interpretation:
    \begin{enumerate}[label=\Alph*.]
        \item Since $p < \alpha$, the data would be unusual if there were truly no mean difference.
        \item Since $p < \alpha$, the true after-level is 0.
        \item Since $p < \alpha$, the data would not be unusual under no difference.
        \item Since $p > \alpha$, there’s no change.
        \item Therefore mindfulness should be mandatory.
    \end{enumerate}
\item Which model is used?
    \begin{enumerate}[label=\Alph*.]
        \item Normal model with paired data
        \item Normal model with two independent groups
        \item Chi-squared test
        \item Linear regression
        \item Binomial model
    \end{enumerate}
\item In this context, power is
    \begin{enumerate}[label=\Alph*.]
        \item $\Pr(H_0\ \text{true})$
        \item $\Pr(\text{reject }H_0 \mid H_0\ \text{true})$
        \item $\Pr(\text{reject }H_0 \mid H_A\ \text{true})$
        \item Same as the p-value
        \item $\Pr(p<0.05)$ without context
    \end{enumerate}
\item Which definitely increases power?
    \begin{enumerate}[label=\Alph*.]
        \item Setting \texttt{paired = TRUE} in any test
        \item Increasing $\alpha$ and decreasing $n$
        \item Decreasing $\alpha$
        \item Repeating identical tests without more data
        \item Increasing the sample size
    \end{enumerate}
\textbf{Information for Questions 31--33 (Two independent groups)}\\
Two drugs are compared on blood pressure reduction. Group 1 (Drug A): $\bar y_1=23.4$, $s_1=5.8$, $n_1=45$. Group 2 (Drug B): $\bar y_2=20.1$, $s_2=6.2$, $n_2=40$.
\item The estimated standard error of $\bar y_1-\bar y_2$ is closest to
    \begin{enumerate}[label=\Alph*.]
        \item 0.95
        \item 1.10
        \item 1.31
        \item 1.60
        \item 2.00
    \end{enumerate}
\item A two-sided test of equal means gives $t\approx 2.52$ with $p \approx 0.013$. At $\alpha=0.05$, we should
    \begin{enumerate}[label=\Alph*.]
        \item Fail to reject; no evidence of a difference.
        \item Reject; evidence of a difference.
        \item Conclude Drug B is superior.
        \item Conclude both drugs are identical in effect.
        \item Need a paired test instead.
    \end{enumerate}
\item A plausible 95\% CI for $\mu_1-\mu_2$ is
    \begin{enumerate}[label=\Alph*.]
        \item $(-5.9,\ -0.7)$
        \item $(-0.7,\ 5.9)$
        \item $(0.7,\ 5.9)$
        \item $(1.3,\ 6.5)$
        \item $(3.3,\ 6.0)$
    \end{enumerate}

\textbf{Information for Questions 34--36 (Proportions)}\\
In $n=150$ visitors, $x=45$ purchased.
\item A 95\% CI for the proportion $p$ should be
    \begin{enumerate}[label=\Alph*.]
        \item $\bar y \pm t\,s/\sqrt n$
        \item $\hat p \pm z\,\sqrt{\hat p(1-\hat p)/n}$
        \item $\hat p \pm t\,\sqrt{\hat p(1-\hat p)/n}$
        \item $\mu \pm z\,\sigma/\sqrt n$
        \item Not possible without $\sigma$
    \end{enumerate}
\item If we increase $n$ from 150 to 600 (same $\hat p$), the CI width
    \begin{enumerate}[label=\Alph*.]
        \item Doubles
        \item Stays the same
        \item Halves
        \item Becomes four times wider
        \item Is unpredictable
    \end{enumerate}
\item A 90\% CI for $p$ is $(0.22,\ 0.36)$. Best interpretation:
    \begin{enumerate}[label=\Alph*.]
        \item 90\% of individuals have outcomes in this range.
        \item There’s a 90\% chance the random future sample proportion lies in this range.
        \item We are 90\% confident the true population proportion lies between 0.22 and 0.36.
        \item 90\% of possible samples will produce $\hat p$ equal to 0.29.
        \item $p$ varies from study to study.
    \end{enumerate}

\textbf{Normal tools in R}\\
\item Which gives $\Pr(Z>1.2)$ for $Z\sim\mathcal N(0,1)$?
    \begin{enumerate}[label=\Alph*.]
        \item pnorm(1.2)
        \item pnorm(-1.2)
        \item 1 - pnorm(1.2)
        \item 1 - pnorm(-1.2)
        \item pnorm(0)
    \end{enumerate}
\item Which gives the 97.5th percentile of $Z\sim\mathcal N(0,1)$?
    \begin{enumerate}[label=\Alph*.]
        \item qnorm(0.025)
        \item qnorm(0.975)
        \item pnorm(0.975)
        \item 1 - qnorm(0.975)
        \item qnorm(1.975)
    \end{enumerate}
\item Which best describes a random sample?
    \begin{enumerate}[label=\Alph*.]
        \item Every unit is selected with equal probability and independently (or via a valid random design).
        \item The most convenient units are selected.
        \item Units chosen until results look ``about right''.
        \item Units with the largest values are over-sampled deliberately (without model).
        \item Volunteers only.
    \end{enumerate}
\item Which action reduces the standard error of $\bar Y$?
    \begin{enumerate}[label=\Alph*.]
        \item Halving $n$
        \item Doubling $n$
        \item Quadrupling $n$
        \item Doubling $s$
        \item None of the above
    \end{enumerate}
\item Which is a parameter, not a statistic?
    \begin{enumerate}[label=\Alph*.]
        \item Sample mean sodium $=3.4$ mg
        \item Sample sd $=1.1$ mg
        \item Population mean sodium $\mu$
        \item $\bar y$ from a study of 40 people
        \item The sample median
    \end{enumerate}
\item Suppose $\sigma=8$, $n=64$, $\bar y=102$, $H_0:\mu=100$. The two-sided $p$-value equals
    \begin{enumerate}[label=\Alph*.]
        \item pnorm(2)
        \item $2 * (1 - \text{pnorm}(2))$
        \item $1 - \text{pnorm}(2)$
        \item pnorm(-2)
        \item $2 * \text{pnorm}(2)$
    \end{enumerate}
\item When should a paired $t$-test be preferred over a two-sample $t$-test?
    \begin{enumerate}[label=\Alph*.]
        \item When two groups are independent.
        \item When measurements are ``before–after'' on the same units.
        \item When one group is twice as large.
        \item When variances are equal.
        \item When sample sizes are both $\ge 30$.
    \end{enumerate}
\item Even if the population is not normal, the distribution of $\bar Y$ is approximately normal when
    \begin{enumerate}[label=\Alph*.]
        \item $n$ is large (and the population isn’t extremely skewed).
        \item $s=0$.
        \item $n=2$.
        \item We take medians instead.
        \item Only if the population is uniform.
    \end{enumerate}
\item For $Y\sim\mathcal N(100,15^2)$, which R command gives the 90th percentile?
    \begin{enumerate}[label=\Alph*.]
        \item qnorm(0.1, 100, 15)
        \item pnorm(0.9, 100, 15)
        \item qnorm(0.9, mean=100, sd=15)
        \item 1 - qnorm(0.9, 100, 15)
        \item qnorm(1.9, 100, 15)
    \end{enumerate}
\item Which statement about the sampling distribution of $\bar Y$ is correct?
    \begin{enumerate}[label=\Alph*.]
        \item Mean is $\mu$, standard error is $\sigma\sqrt n$.
        \item Mean is $\bar y$, standard error is $s/\sqrt n$.
        \item Mean is $\mu$, standard error is $\sigma/\sqrt n$.
        \item Mean is 0, standard error is 1.
        \item It is never approximately normal.
    \end{enumerate}
\end{enumerate}

\newpage
\section*{Solutions}
\begin{enumerate}
\item B. 14.5. $\bar y=(12.5+15.2+14.8+13.9+16.1)/5=72.5/5=14.5$.
\item C. 1.4. Compute $s$ (unbiased, $n-1$ in denominator); $s\approx 1.37$, closest is 1.4.
\item C. We need the distribution (range/variability) of times to judge “unusual”.
\item A. Models describe variability; we quantify uncertainty (confidence, p-values, etc.).
\item D. 0.50+0.30=0.80.
\item C. $E[Y]=1(0.20)+2(0.50)+5(0.30)=0.2+1.0+1.5=2.7$.
\item B. It’s the long-run average under the distribution.
\item A. Probability a non-subscriber purchases the premium plan.
\item D. $\Pr(A\mid B)=\dfrac{0.35\cdot 0.40}{0.35\cdot 0.40+0.10\cdot 0.60}=0.14/0.20=0.70$.
\item C. $\Pr(A\cup B)=\Pr(A)+\Pr(B)-\Pr(A\cap B)=0.40+0.20-0.14=0.46$.
\item B. $\Pr(B)=0.35(0.40)+0.10(0.60)=0.14+0.06=0.20$.
\item C. A random process with a numerical outcome.
\item C. CO$_2$ concentration is continuous (in practice, measured on a continuum).
\item D. $E[3X-2Y]=3(8)-2(5)=24-10=14$.
\item C. $\mathrm{Var}(3X-2Y)=9(1.2)+4(0.8)=10.8+3.2=14$; $sd=\sqrt{14}\approx 3.74$.
\item C. $y=\mu+z\sigma=70+(-0.75)\cdot 12=70-9=61$.
\item B. $z=(82-70)/12=1.0$; $\Pr(Y>82)=1-\Phi(1.0)= 1 - \text{pnorm}(1.0)$.
\item B. $\bar Y\sim\mathcal N(70,\ (12/\sqrt{100})^2) \Rightarrow sd =1.2$.
\item B. By CLT, approximately normal with mean 70, sd $12/\sqrt{100}=1.2$.
\item E. With unknown $\sigma$, smallish $n=25$, use $t$: $\bar y \pm t_{24,0.975}\,s/\sqrt n$.
\item D. Only the multiplier changes (larger), making the interval wider.
\item E. Confidence refers to the true mean lying in the interval.
\item C. A tiny $p$-value indicates data are very unlikely under $H_0$.
\item E. Estimation uses a statistic (e.g., $\bar y$) to infer a parameter (e.g., $\mu$).
\item C. The sample mean of weight.
\item B. The sample standard deviation of length.
\item A. Since $p=0.012<0.05$, data would be unusual if there were truly no difference.
\item A. Paired $t$-test (normal model for paired differences).
\item C. Power $=\Pr(\text{reject }H_0\mid H_A\text{ true})$.
\item E. Increasing sample size increases power (all else equal).
\item C. $\mathrm{SE}=\sqrt{s_1^2/n_1+s_2^2/n_2}=\sqrt{5.8^2/45+6.2^2/40}\approx 1.31$.
\item B. With $p\approx 0.013<0.05$, reject $H_0$; there is evidence of a difference.
\item C. $95\%$ CI $\approx 3.3 \pm 1.96\times 1.31 \approx (0.7,5.9)$.
\item B. For a single proportion: $\hat p \pm z\,\sqrt{\hat p(1-\hat p)/n}$.
\item C. Width scales with $1/\sqrt n$. $n \times 4 \Rightarrow$ width $\times 1/2$.
\item C. Confidence speaks to the population proportion.
\item C. $\Pr(Z>1.2)=1-\Phi(1.2)= 1 - \text{pnorm}(1.2)$.
\item B. 97.5th percentile is $\text{qnorm}(0.975)$.
\item A. Random sampling requires a legitimate random mechanism (independence/equal chance).
\item C. Quadrupling $n$ halves $s/\sqrt n$.
\item C. $\mu$ is a parameter (population quantity).
\item B. $z=(102-100)/(8/\sqrt{64})=2$; two-sided $p =2(1-\Phi(2))$.
\item B. Paired test for before–after on the same units.
\item A. CLT: for large $n$, $\bar Y$ approximately normal (non-extreme skewness).
\item C. qnorm(0.9, mean=100, sd=15).
\item C. Mean $\mu$, standard error $\sigma/\sqrt n$.
\end{enumerate}
\end{document}
