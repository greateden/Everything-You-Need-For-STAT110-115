\documentclass[12pt]{book}

\usepackage[a4paper,top=3cm,bottom=2cm,left=2cm,right=2cm,marginparwidth=1.2cm]{geometry}
\usepackage{fancyhdr}


\author{Eden Li}
\title{STAT110/115 Tutoring Materials – 01 Terms}
\date{}

\begin{document}
\pagestyle{fancy}
\lhead{STAT110/115 Tutoring Materials }
\rhead{Terms}
%\maketitle


\begin{itemize}
\item $\mu$: Population mean (mu).
\item $\sigma$: Population standard deviation (sigma).
\item $\pi$: Population proportion.
\item $\hat{\mu}$: An estimate of the parameter $\mu$.
\item $\bar{x}$: A statistic and an estimate (for STAT110/115 paper only).
\item Proportion: Fraction of one quantity when compared to the whole.
\item Ratio: Fraction given by one quantity over another. \\
\textbf{N.B.}, both quantities have the same units.
\item Rates (the difference between rates and ratio): Rates are like ratios for quantities with different units.
\item Random variables are described by: Probability distributions.
\item Observed values of random variables are: Data.
\item Difference between random variables and observed/realised value: \\
Random variables: unknown quantity varies unpredictably; \\
Observed/realised value: got the actual quantity of the unknown quantity.
\item Types of variables: 
\begin{itemize}
\item Continuous - can be expressed on a continuous scale in which every value is possible.
\item Discrete - can be in one-to-one correspondence with the counting numbers.
\item Categorical - restricted to one of a set of categories. For example 'Heads' or 'Tails'.
\begin{itemize}
\item type 1: 0 - 1 binary, A/B/O/AB more than two
\item type 2: A/B/O/AB nominal, pass/fail ordinal
\end{itemize}
\end{itemize}
\item Types of censored data: Right censored, Left censored, Interval-censored \\
\textbf{N.B.}, censored data are categorised by two variables: the censor type and the censoring point or interval.
\begin{itemize}
\item Right censored: The true value is known to be larger than a recorded value. \\
e.g., we know that someone lived until at least 31 Dec 2017. 50+
\item Left censored: The true value is known to be smaller than a recorded value. \\
e.g., we know that a measurement is less than a known limit of detection. 10-
\item Interval-censored: The true value is known to lie between two values. \\
e.g., we know the date of infection with HPV is after a negative test and before a positive test 2 years later.
\end{itemize}
\end{itemize}
\end{document}